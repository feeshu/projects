\documentclass[12pt,a4paper]{article}
\usepackage{amsmath}
\usepackage{indentfirst}
\usepackage[pdftex]{graphicx}
\usepackage{epstopdf}
\usepackage[a4paper,bindingoffset=0.2in,left=.75in,right=.75in,top=1in,bottom=1in,footskip=.25in]{geometry}
\usepackage{amssymb}


\newcommand{\HRule}{\rule{\linewidth}{0.5mm}}


\begin{document}

\begin{titlepage}
\begin{center}

\includegraphics[width=100pt, height=100pt]{C:/Users/Alex/Desktop/univ}~\\[1cm]

\textsc{\LARGE Universitatea din Bucuresti}\\[1.5cm]
\textsc{\Large Proiect Probabilitati si Statistica}\\[0.5cm]

\HRule \\[0.4cm]
{ \huge \bfseries Distributii si densitati de repartitie \\[0.4cm]}
\HRule \\[1.5cm]

\begin{minipage}{0.4\textwidth}
\begin{flushleft} \large
\emph{Student:}\\
Alexandru \textsc{Marin}
\end{flushleft}
\end{minipage}
\begin{minipage}{0.4\textwidth}
\begin{flushright} \large

\end{flushright}
\end{minipage}

\vfill

{\large \begin{center}
10 Decembrie, 2013\end{center}}

\end{center}
\end{titlepage}

\begin{center}
\section{\underline{\textbf{DENSITATI DE REPARTITIE}}}
\end{center}

\subsection{\textbf{Distributia uniforma}}
\subsubsection{\textbf{Densitatea si functia de repartitie}}
O variabila aleatoare $\xi$ are o distributie uniforma daca poate lua echiprobabil orice valoare intr-un interval (variabila este uniform distribuita in interval), adica f(x) = ct, cu x$\in$[a,b]. Variabilele aleatoare pot fi considerate ca fiind functii de la un domeniu oarecare catre multimea numerelor reale.\\
\-\hspace{0.4cm} Daca a si b sunt capetele intervalului, atunci \textbf{densitatea de probabilitate} va fi:\\

\begin{center}
$\omega_\xi(x) = \begin{cases}
\frac{1}{b-a}, & a \le x \le b \\
0, & \text{in rest}
\end{cases}$
\end{center}

Valoarea medie a unei variabile de genul acesta este $\frac{a+b}{2}$.\\\\
Pentru a vizualiza graficul densitatii, vom alege intervalul [1,5].\\
Astfel, pentru:
\begin{itemize}
\renewcommand{\labelitemi}{$\bullet$}
\item $x=2,  \omega_\xi(2) = \frac{1}{5-1} = \frac{1}{4}$;
\item $x=3,  \omega_\xi(3) = \frac{1}{5-1} = \frac{1}{4}$;
\item $x=4,  \omega_\xi(4) = \frac{1}{5-1} = \frac{1}{4}$;
\end{itemize}


\begin{center}
	\includegraphics[scale=0.5]{C:/Users/Alex/Desktop/statistica/unifpdf.eps}
\end{center}
{\scriptsize Cod \textbf{Matlab}:\\
\fbox{\parbox{0.8\linewidth}{%
		x  = [1:1:5];\\y = unifpdf(x,1,5);\\plot(x,y)}}}~\\[1cm]
Forma distributiei indica faptul ca o variabila aleatoare uniforma poate lua cu aceeasi probabilitate orice valoare in intervalul [a,b], dar nu poate lua nicio valoare in exteriorul acestuia.\\\\
\textbf{Functia de repartitie} a distributie uniforme: $F(x|a,b)=\frac{x-a}{b-a}I_|a,b|(x)$.

\begin{center}
	\includegraphics[scale=0.5]{C:/Users/Alex/Desktop/statistica/unifcdf.eps}~\\[0.5cm]
\end{center}
{\scriptsize Cod \textbf{Matlab}:\\
\fbox{\parbox{0.8\linewidth}{%
		x  = [0:1:6];\\y = unifcdf(x,1,5);\\plot(x,y)}}}~\\[1cm]
Se poate observa faptul ca probabilitatea ca variabila \emph{x} sa ia valori mai mari creste liniar in intervalul [1,5], in afara acestuia functia luand valoarea 0, pentru $x < a$, respectiv 1, pentru $x \geq b$. 

\subsubsection{\textbf{Dependenta de parametri}}
Pentru a determina dependenta distributiei de parametrii a si b, vom schimba, pe rand, fiecare parametru si vom studia graficul.\\
	\includegraphics[scale=0.5]{C:/Users/Alex/Desktop/statistica/paramunifpdf.eps}~\\
{\scriptsize Cod \textbf{Matlab}:\\
\fbox{\parbox{0.8\linewidth}{%
		x  = [3:1:8];\\a = unifpdf(x,2,8);\\b = unifpdf(x,2,9);\\c = unifpdf(x,2,10);\\plot(x,a,'green',x,b,'red',x,c,'blue')}}}~\\[1cm]
 	In cazul \emph{b = variabil}, distributia isi pastreaza proprietatea, astfel variabila \emph{x} poate lua diferite valori cu aceeasi probabilitate, insa probabilitatea scade odata cu cresterea limitei drepte a intervalului (\emph{punctul b}).\\\\
	\includegraphics[scale=0.5]{C:/Users/Alex/Desktop/statistica/param2unifpdf.eps}~\\
{\scriptsize Cod \textbf{Matlab}:\\
\fbox{\parbox{0.8\linewidth}{%
		x  = [3:1:8];\\a = unifpdf(x,3,8);\\b = unifpdf(x,2,8);\\c = unifpdf(x,1,8);\\plot(x,a,'red',x,b,'green',x,c,'blue')}}}~\\[1cm]
	Spre deosebire de cazul precedent, in acest caz probabilitatea creste odata cu scaderea limitei stangi a intervalului (\emph{punctul a}).\\\\
Pentru valori aleatorii ale variabilei x vom folosi functia \emph{unifrnd} pentru parametrii precedenti si vom reprezenta grafic histogramele valorilor.\\\\
	 \includegraphics[scale=0.5]{C:/Users/Alex/Desktop/statistica/unifrnd.eps}~\\
{\scriptsize Cod \textbf{Matlab}:\\
\fbox{\parbox{0.8\linewidth}{%
		x  = unifrnd(2,8,1,1000);\\y = unifrnd(2,9,1,1000);\\z = unifrnd(2,10,1,1000);\\ a = unifrnd(1,8,1,1000);\\b = unifrnd(2,8,1,1000);\\c = unifrnd(3,8,1,1000);\\subplot(1,2,1);\\hist(x)\\hold on\\hist(y)\\hist(z)\\subplot(1,2,2)\\hist(c)\\hold on\\hist(b)\\hist(a)}}}~\\[1cm]
	Putem considera fluctuatia probabilitatii neglijabila, drept urmare \emph{x} poate lua cu aproximativ aceeasi probabilitate orice valoare din intervalul specificat, indiferent de parametri.\\

\subsection{\textbf{Distributia exponentiala}}
\subsubsection{\textbf{Densitatea si functia de repartitie}}
O variabila continua \emph{X} are distributie exponentiala de parametru $\lambda > 0$ daca are densitatea de probabilitate de forma:\\
\begin{equation*}
f(x)=\lambda \cdot e^{-\lambda \cdot x}, x \geq 0
\end{equation*}, unde $\lambda = \frac{1}{\mu}$.\\
Valoarea medie a acestei variabile este $\frac{1}{\lambda}$.\\\\
Pentru $\mu = 2$, $f(x|\mu=2)=\frac{1}{2}\cdot e^{-\frac{x}{2}}$.\\
Pentru $\mu = 3$, $f(x|\mu=3)=\frac{1}{3}\cdot e^{-\frac{x}{3}}$.\\\\
	\includegraphics[scale=0.5]{C:/Users/Alex/Desktop/statistica/exppdf.eps}~\\
{\scriptsize Cod \textbf{Matlab}:\\
\fbox{\parbox{0.8\linewidth}{%
		x  = [0:0.1:20];\\y = exppdf(x,2);\\plot(x,y)}}}~\\\\\\
Forma distributiei indica faptul ca probabilitatea ca \emph{x} sa ia valori mici este ridicata, iar probabilitatea ca variabila sa ia valori mici este foarte scazuta.\\\\\\
\textbf{Functia de repartitie} a distributiei exponentiale este: $F(x|\mu) = \int_{0}^{x}\frac{1}{\mu}e^{\frac{-t}{\mu}}dt = 1-e^{\frac{-x}{\mu}}$ .\\
	\includegraphics[scale=0.5]{C:/Users/Alex/Desktop/statistica/expcdf.eps}~\\
{\scriptsize Cod \textbf{Matlab}:\\
\fbox{\parbox{0.8\linewidth}{%
		x  = [0:0.1:20];\\y = expcdf(x,2);\\z = expcdf(x,3.5);\\a = expcdf(x,5); \\plot(x,y,'r',x,z,'g',x,a,'b')}}}~\\\\
	Functia creste exponential si reiese faptul ca x poate lua cu probabilitate ridicata valori mari, in timp ce pentru valori mici exista o probabilitate scazuta. \\
\subsubsection{\textbf{Dependenta de parametri}}
Pentru a demonstra dependenta functiei de parametrul $\mu$, vom alege valori diferite si vom reprezenta grafic functia.\\\\
\includegraphics[scale=0.5]{C:/Users/Alex/Desktop/statistica/paramexp.eps}~\\[2cm]
{\scriptsize Cod \textbf{Matlab}:\\
\fbox{\parbox{0.8\linewidth}{%
		x  = [0:0.1:20];\\y = exppdf(x,0.5);\\z = exppdf(x,1.5);\\ a = exppdf(x,2.5);\\plot(x,y,'r',x,z,'g',x,a,'b')}}}~\\[1cm]
Din grafic reiese ca, odata cu reducerea valorii lui $\mu$, probabilitatea ca \emph{x} sa ia valori mici scade, in timp ce probabilitatea ca \emph{x} sa ia valori mari ramane la fel de scazuta.\\\\
Pentru valori aleatorii ale variabilei \emph{x} vom folosi functia \emph{exprnd} cu parametrii precedenti si vom reprezenta grafic histogramele.\\\\
	\includegraphics[scale=0.5]{C:/Users/Alex/Desktop/statistica/exprnd.eps}~\\
{\scriptsize Cod \textbf{Matlab}:\\
\fbox{\parbox{0.8\linewidth}{%
		x  = [0:0.1:20];\\y = exprnd(0.5,1,1000);\\z = exprnd(1.5,1,1000);\\a = exprnd(2.5,1,1000);\\hist(a)\\hold on\\hist(z)\\hist(y)}}}~\\\\	
Dupa cum stiam deja din formula densitatii, valorile mici sunt cele mai probabile, valorile mari avand o probabilitate foarte scazuta.\\

\subsection{\textbf{Distributia normala}}
Distributia normala este o distributie de probabilitate continua. Este numita de asemenea \textbf{distributia Gauss}, deoarece a fost descoperita de catre \textbf{Carl Friedrich Gauss}.\\
Distributia normala standard(cunoscuta si sub numele de \textbf{distributie Z}) este distributia normala cu media zero si variatia 1. Aceasta este adesea numita \textbf{clopotul lui Gauss}, deoarece graficul densitatii de probabilitate arata ca un clopot.\\
Se noteaza cu $N(\mu,\sigma)$, unde $\mu$ si $\sigma$ sunt parametrii din functia de distributie care va fi descrisa in continuare.\\
\begin{equation*}
f(x|\mu,\sigma) = \frac{1}{\sigma\sqrt{2\pi}}e^{\frac{-(x-\mu)^{2}}{2\sigma^{2}}}
\end{equation*}
$f(x|\mu=0,\sigma^{2}=5) = \frac{1}{2.23\sqrt{2\pi}}e^{\frac{-(x)^{2}}{10}}$\\
$f(x|\mu=-5,\sigma^{2}=0.2) = \frac{1}{0.44\sqrt{2\pi}}e^{\frac{-(x+5)^{2}}{0.4}}$\\\\
\includegraphics[scale=0.5]{C:/Users/Alex/Desktop/statistica/normpdf.eps}~\\
{\scriptsize Cod \textbf{Matlab}:\\
\fbox{\parbox{0.8\linewidth}{%
		x  = [-10:0.1:10];\\y = normpdf(x,0,2.23);\\plot(x,y)}}}~\\\\
Densitatea este simetrica in jurul punctului $\mu$ (in cazul nostru $\mu=0$), avand probabilitatea foarte mare de a lua valoarea 0, scazand simetric in afara acestei valori.\\\\
\textbf{Functia de repartitie} a distributiei normale este: 
\begin{equation*}
F(X|\mu,\sigma) = \frac{1}{\sigma\sqrt{2\pi}}\int_{-\infty}^{x}e^{\frac{-(t-\mu)^{2}}{2\sigma^{2}}}dt
\end{equation*}
\includegraphics[scale=0.5]{C:/Users/Alex/Desktop/statistica/normcdf.eps}~\\
{\scriptsize Cod \textbf{Matlab}:\\
\fbox{\parbox{0.8\linewidth}{%
		x  = [-10:0.1:10];\\y = normcdf(x,0,2);\\z = normcdf(x,0,3.5)\\ a = normcdf(x,0,5)\\ b = normcdf(x,-5,1)\\plot(x,y,'r',x,z,'g',x,a,'b',x,b,'br')}}}~\\\\
Functia variaza in jurul punctului $\mu$; pentru valori $a<\mu$ probabilitatea ca x sa ia valorile a scade, iar pentru valori $b>\mu$ probabilitatea creste.

\subsubsection{\textbf{Dependenta de parametri}}
Pentru a demonstra dependenta distributiei de parametrii $\mu$ si $\sigma$, vom lua valori aleatorii pentru acestea si vom reprezenta grafic distributia.\\\\

\includegraphics[scale=0.5]{C:/Users/Alex/Desktop/statistica/paramnormcdf.eps}~\\
{\scriptsize Cod \textbf{Matlab}:\\
\fbox{\parbox{0.8\linewidth}{%
		x  = [-10:0.1:10];\\y = normpdf(x,1,0.7);\\z = normpdf(x,1,1.5);\\a = normpdf(x,1,2.6);\\plot(x,y,'r',x,z,'g',x,a,'b')}}}~\\\\
Odata cu cresterea variabilei $\sigma$ creste si dispersia functiei, \emph{x} avand probabilitatea mai mare de a lua valori in jurul punctului $\mu$.\\\\

\includegraphics[scale=0.5]{C:/Users/Alex/Desktop/statistica/param2normcdf.eps}~\\[2cm]
{\scriptsize Cod \textbf{Matlab}:\\
\fbox{\parbox{0.8\linewidth}{%
		x  = [-10:0.1:10];\\y = normpdf(x,-5,1.5);\\z = normpdf(x,-1,1.5)\\ a = normpdf(x,3,1.5)\\plot(x,y,'r',x,z,'g',x,a,'b')}}}~\\\\ 
In acest caz, centrul functiei se deplaseaza odata cu schimbarea variabilei $\mu$, astfel pentru valori mai mici, functia se deplaseaza spre stanga axei, in timp ce dispersia ramane neschimbata.\\\\
Vom alege valori aleatorii pentru variabila x folosind functia \emph{normrnd} si vom reprezenta grafic histogramele.\\\\
\includegraphics[scale=0.5]{C:/Users/Alex/Desktop/statistica/normrnd.eps}~\\[1cm]
{\scriptsize Cod \textbf{Matlab}:\\
\fbox{\parbox{0.8\linewidth}{%
		x = normrnd(1,0.7,1,1000);\\y = normrnd(1,1.5,1,1000);\\z = normrnd(1,2.6,1,1000);\\a = normrnd(-5,1.5,1,1000);\\b = normrnd(-1,1.5,1,1000);\\c = normrnd(3,1.5,1,1000);\\subplot(1,2,1)\\hist(z)\\hold on\\hist(y)\\hist(x)\\subplot(1,2,2)\\hist(c)\\hold on\\hist(b)\\hist(a)}}}~\\\\ 
Si in cazul valorilor aleatorii se respecta proprietatile enumerate mai sus.\\\\

\subsection{\textbf{Distributia Pareto generalizata}}
\subsubsection{\textbf{Densitatea si functia de repartitie}}
In statistica, \textbf{distributia Pareto generalizata (GPD)} reprezinta o familie de distributii de probabilitate continue. Este specificata de trei parametri: forma k ($k \neq 0$), scala $\sigma$ si limita $\theta$.\\
\begin{equation*}
f(x|k,\sigma,\theta)=\left(\frac{1}{\sigma}\right)\left(1+k\frac{(x-\theta)}{\sigma}\right)^{-1-\frac{1}{k}}
\end{equation*}
$f(x|2,4,1)=\left(\frac{1}{4}\right)\left(1+\frac{(x-1)}{2}\right)^{-\frac{3}{2}}$\\
$f(x|3,3,3)=\left(\frac{1}{3}\right)\left(1+(x-3)\right)^{-\frac{4}{3}}$\\\\
\includegraphics[scale=0.5]{C:/Users/Alex/Desktop/statistica/gppdf.eps}~\\[2cm]
{\scriptsize Cod \textbf{Matlab}:\\
\fbox{\parbox{0.8\linewidth}{%
		x  = [0:0.1:20]y = gppdf(x,2,4,1);\\z = gppdf(x,3,3,3);\\plot(x,y,'r',x,z,'g')}}}~\\\\ 
Avand ca punct de plecare punctul limita $\theta$, densitatea scade exponential pana cand variabila \emph{x} va avea o probabilitate foarte scazuta de a primi o valoare mare.\\\\
\textbf{Functia de repartitie} a \textbf{distributiei Pareto generalizate} este:\\
\begin{equation*}
f(x|k,\sigma,\theta)=1-\left(1+k\frac{x-\theta}{\sigma}\right)^{\frac{-1}{k}}
\end{equation*}
\includegraphics[scale=0.5]{C:/Users/Alex/Desktop/statistica/gpcdf.eps}~\\[2cm]
{\scriptsize Cod \textbf{Matlab}:\\
\fbox{\parbox{0.8\linewidth}{%
		x  = [0:0.1:20]y = gpcdf(x,2,4,1);\\z = gpcdf(x,3,3,3);\\plot(x,y,'r',x,z,'g')}}}~\\\\ 
Spre deosebire de densitate, functia de repartitie creste exponential pana in punctul in care variabila \emph{x} va avea o probabilitate foarte mare de a primi valori mari.\\\\
\subsubsection{\textbf{Dependenta de parametri}}
Pentru a studia dependenta de parametri, vom alege, pe rand, valori diferite pentru k, $\sigma$ si $\theta$ si vom reprezenta grafic rezultatele.\\\\
Vom observa ca:\\
\begin{itemize}
\renewcommand{\labelitemi}{$\bullet$}
\item pentru \textbf{k variabil}, cele 3 densitati pornesc cu aceeasi probabilitate, insa odata cu cresterea variabilei k scade dispersia, astfel pentru un k mai mare, probabilitatea ca x sa ia valori mari scade exponential.
\item pentru \textbf{$\sigma$ variabil}, odata cu cresterea variabilei scade probabilitatea densitatii in punctul de plecare. Astfel, pentru un $\sigma$ mai mare, probabilitatea ca \emph{x} sa ia valori mici scade, iar probabilitatea valorilor mari ramane constanta.
\item pentru \textbf{$\theta$ variabil}, difera punctul de plecare pentru cele 3 densitati. Astfel, pentru un $\theta$ mai mare, densitatea isi va muta punctul de plecare spre dreapta, deci va avea probabilitatea mai ridicata de a lua valori mai mari.
\end{itemize}
\includegraphics[scale=0.42]{C:/Users/Alex/Desktop/statistica/gpparam.eps}~\\[0.5cm]
{\scriptsize Cod \textbf{Matlab}:\\
\fbox{\parbox{0.8\linewidth}{%
		y = gppdf(x,1.5,3,1);\\z = gppdf(x,2.5,3,1);\\a = gppdf(x,3.5,3,1);\\b = gppdf(x,1.5,1.5,1);\\c = gppdf(x,1.5,2.5,1);\\d = gppdf(x,1.5,3.5,1);\\e = gppdf(x,1.5,2.5,2);\\f = gppdf(x,1.5,2.5,3);\\g = gppdf(x,1.5,2.5,4);\\subplot(2,2,1)\\plot(x,y,'r',x,z,'g',x,a,'b')\\subplot(2,2,2)\\plot(x,b,'r',x,c,'g',x,d,'b')\\subplot(2,2,3)\\plot(x,e,'r',x,f,'g',x,g,'b')}}}~\\\\ 
Pentru a evidentia densitatea, vom alege valori aleatorii pentru variabila \emph{x} folosind functia \emph{gprnd} si vom reprezenta grafic dispersia acestora.\\
Astfel, vom observa ca valorile foarte mici (0-0.5) sunt foarte probabile, in timp ce valorile mai mari au o probabilitate foarte scazuta, indiferent de parametrii folositi.\\
\includegraphics[scale=0.5]{C:/Users/Alex/Desktop/statistica/gprnd.eps}~\\[0.5cm]
{\scriptsize Cod \textbf{Matlab}:\\
\fbox{\parbox{0.8\linewidth}{%
		x=gprnd(1.5,3,1,1,1000);\\y=gprnd(2.5,3,1,1,1000);\\z=gprnd(3.5,3,1,1,1000);\\a=gprnd(1.5,1.5,1,1,1000);\\b=gprnd(1.5,2.5,1,1,1000);\\c=gprnd(1.5,3.5,1,1,1000);\\d=gprnd(1.5,2.5,2,1,1000);\\e=gprnd(1.5,2.5,3,1,1000);\\f=gprnd(1.5,2.5,4,1,1000);\\subplot(2,2,1)\\hist(y)\\hold on\\hist(z)\\subplot(2,2,2)\\hist(a)\\hold on\\hist(b)\\hist(c)\\subplot(2,2,3)\\hist(f)\\hold on\\hist(e)}}}~\\\\

\subsection{\textbf{Distributia chi-patrat ($\chi^{2}$)}}
\subsubsection{\textbf{Densitatea si functia de repartitie}}
In teoria probabilitatilor si statistica, \textbf{distributia chi-patrat} cu \emph{v} grade de libertate reprezinta distributia unei sume de patrate ale \emph{v} variabile aleatorii, independente si normale. Densitatea distributiei este:
\begin{equation*}
f(x|v)=\frac{x^{(v-2)/2}e^{-x/2}}{2^{v/2}\Gamma(v/2)}
\end{equation*}
unde $\Gamma$( $\cdot$ ) este functia Gamma, iar v,x $\geq$ 0.\\\\\\
\textbf{Exemplu:}\\
$f(x|8) = \frac{x^{3}e^{-4}}{16\cdot 6} = \frac{x^{3}e^{-4}}{96}$, pentru $\Gamma(n)=(n-1)!$\\
$f(x|10) = \frac{x^{4}e^{-5}}{32\cdot 24} = \frac{x^{4}e^{-5}}{768}$\\\\
\includegraphics[scale=0.5]{C:/Users/Alex/Desktop/statistica/chi2pdf.eps}~\\[0.5cm]
{\scriptsize Cod \textbf{Matlab}:\\
\fbox{\parbox{0.8\linewidth}{%
x=[0:0.1:20];\\y=chi2pdf(x,8);\\z=chi2pdf(x,10);\\plot(x,y,'r',x,z,'g');}}}~\\\\
Se poate observa faptul ca probabilitatea ca \emph{x} sa ia valori foarte mici sau foarte mari este foarte scazuta, probabilitatea cea mai mare fiind pentru valorile intermediare.\\
\textbf{Functia de repartitie} a distributiei chi-patrat:\\
\begin{equation*}
F(x|v)=\int_{0}^{x}\frac{t^{(v-2)/2}e^{-t/2}}{2^{v/2}\Gamma(v/2)}dt
\end{equation*}
unde $\Gamma$( $\cdot$ ) este functia Gamma.\\\\
Forma functiei de repartitie ne indica faptul ca \emph{x} poate lua valori mici cu o probabilitate scazuta, in timp ce pentru valori mari probabilitatea este ridicata.\\
\includegraphics[scale=0.5]{C:/Users/Alex/Desktop/statistica/chi2cdf.eps}~\\[0.5cm]
{\scriptsize Cod \textbf{Matlab}:\\
\fbox{\parbox{0.8\linewidth}{%
x=[0:0.1:20];\\y=chi2cdf(x,8);\\z=chi2cdf(x,10);\\plot(x,y,'r',x,z,'g');}}}~\\\\
\subsubsection{\textbf{Dependenta de parametri}}
Pentru a demonstra dependenta densitatii de parametrul \emph{v}, vom alege valori diferite pentru acesta si vom reprezenta grafic diferentele.\\\\
\includegraphics[scale=0.5]{C:/Users/Alex/Desktop/statistica/chi2param.eps}~\\[0.5cm]
{\scriptsize Cod \textbf{Matlab}:\\
\fbox{\parbox{0.8\linewidth}{%
x=[0:0.1:20];\\y=chi2pdf(x,3.5);\\z=chi2pdf(x,6.4);\\a=chi2pdf(x,8);\\b=chi2pdf(x,10.5);\\plot(x,y,'r',x,z,'g',x,a'b',x,b,'m');}}}~\\\\
Cu cat variabila \emph{v} creste, cu atat mai mult creste dispersia functiei, \emph{x} avand astfel probabilitatea de a lua valori dintr-un interval mai mare.\\\\
Pentru a generaliza proprietatile densitatii, vom folosi functia \emph{chi2rnd} pentru a genera valori aleatorii, apoi le vom reprezenta grafic.\\\\
\includegraphics[scale=0.5]{C:/Users/Alex/Desktop/statistica/chi2rnd.eps}~\\[0.5cm]
{\scriptsize Cod \textbf{Matlab}:\\
\fbox{\parbox{0.8\linewidth}{%
a = chi2rnd(3.5,1,1000);\\b = chi2rnd(6.4,1,1000);\\c = chi2rnd(8,1,1000);\\d = chi2rnd(10,1,1000);\\hist(a)\\hold on\\hist(b)\\hist(c)\\hist(d)}}}~\\\\
Se verifica astfel faptul ca odata cu \emph{v} creste si probabilitatea ca \emph{x} sa ia valori din ce in ce mai mari.\\\\

\subsection{\textbf{Distributia Gamma}}
\subsubsection{\textbf{Densitatea si functia de repartitie}}
In domeniul probabilitatilor si statisticii, \textbf{distributia Gamma} este o familie de distributii continue cu doi parametri, unul de forma (\emph{k}) si unul de scala (\emph{$\theta$}). Densitatea de repartitie :
\begin{equation*}
f(x|k,\theta)=\frac{1}{\theta^{k}\Gamma(k)}x^{k-1}e^{\frac{-x}{\theta}}
\end{equation*}
pentru x,k,$\theta \geq$ 0.
Exemplu:\\\\
$f(x|3,2)=\frac{1}{16}x^{2}e^{\frac{-x}{2}}$\\
$f(x|5,1)=\frac{1}{6}x^{4}e^{-x}$\\\\
\includegraphics[scale=0.5]{C:/Users/Alex/Desktop/statistica/gampdf.eps}~\\[0.5cm]
{\scriptsize Cod \textbf{Matlab}:\\
\fbox{\parbox{0.8\linewidth}{%
x = [0:0.1:20];\\y = gampdf(x,3,2);\\z = gampdf(x,5,1);\\plot(x,y,'r',x,z,'g')}}}~\\\\
Forma densitatii ne indica faptul ca valorile mici sunt cele mai probabile, in timp ce probabilitatea valorilor mari scade exponential.\\
\textbf{Functia de repartitie} a distributiei Gamma :
\begin{equation*}
F(x|k,\theta)=\frac{1}{\theta^{k}\Gamma(k)}\int_{0}^{x}t^{k-1}e^{\frac{-t}{\theta}}dt
\end{equation*}
\includegraphics[scale=0.5]{C:/Users/Alex/Desktop/statistica/gamcdf.eps}~\\[0.5cm]
{\scriptsize Cod \textbf{Matlab}:\\
\fbox{\parbox{0.8\linewidth}{%
x = [0:0.1:20];\\y = gamcdf(x,3,2);\\z = gamcdf(x,5,1);\\plot(x,y,'r',x,z,'g')}}}~\\\\
Spre deosebire de densitate, functia de repartitie de indica faptul ca \emph{x} poate lua cu o probabilitate mare valori mari, in timp ce valorile mici au o probabilitate scazuta.\\\\
\subsubsection{\textbf{Dependenta de parametri}}
Pentru a demonstra dependenta de parametri, vom alege valori diferite pentru parametrii \emph{k} si $\theta$ si vom reprezenta grafic rezultatele.\\
\includegraphics[scale=0.5]{C:/Users/Alex/Desktop/statistica/gamparam1.eps}~\\[0.5cm]
{\scriptsize Cod \textbf{Matlab}:\\
\fbox{\parbox{0.8\linewidth}{%
x = [0:0.1:20];a = gampdf(x,2.5,2);\\b = gampdf(x,5,2);\\c = gampdf(x,7.5,2);\\plot(x,a,'r',x,b,'g',x,c,'b')}}}~\\\\
Observam ca, cu cat variabila de forma \emph{k} creste, cu atat creste si dispersia densitatii, astfel probabilitatea ca \emph{x} sa ia valori mari este din ce in ce mai ridicata.\\\\
\includegraphics[scale=0.5]{C:/Users/Alex/Desktop/statistica/gamparam2.eps}~\\[0.5cm]
{\scriptsize Cod \textbf{Matlab}:\\
\fbox{\parbox{0.8\linewidth}{%
x = [0:0.1:20];a = gampdf(x,2,2.5);\\b = gampdf(x,2,5);\\c = gampdf(x,2,7.5);\\plot(x,a,'r',x,b,'g',x,c,'b')}}}~\\\\
In acest caz densitatea se comporta exact ca in primul caz, dispersia crescand odata cu valoarea $\theta$.\\\\
Pentru a generaliza proprietatile densitatii vom alege valori aleatorii pentru parametrii precedenti folosind functia \emph{gamrnd} si vom reprezenta grafic dispersia acestora.\\
\includegraphics[scale=0.5]{C:/Users/Alex/Desktop/statistica/gamrnd.eps}~\\[0.5cm]
{\scriptsize Cod \textbf{Matlab}:\\
\fbox{\parbox{0.8\linewidth}{%
a = gamrnd(2,2.5,1,1000);\\b = gamrnd(2,5,1,1000);\\c = gamrnd(2,7.5,1,1000);\\d = gamrnd(2.5,2,1,1000);\\e = gamrnd(5,2,1,1000);\\f = gamrnd(7.5,2,1,1000);\\subplot(1,2,1)\\hist(c)\\hold on\\hist(b)\\hist(a)\\subplot(1,2,2)\\hist(f)\\hold on\\hist(e)\\hist(d)}}}~\\\\

\subsection{\textbf{Distributia Beta}}
\subsubsection{\textbf{Densitatea si functia de repartitie}}
\textbf{Distributia Beta} este o familie de distributii continue de probabilitate definite pe intervalul [0,1], parametrizate de doi parametri pozitivi de forma, notati $\alpha$ si $\beta$ si care apar ca exponenti ai variabilei aleatoare, controland forma distributiei.\\
Densitatea de repartitiei se defineste prin:\\
\begin{equation*}
f(x|\alpha,\beta)=\frac{x^{\alpha-1}(1-x)^{\beta-1}}{B(\alpha,\beta)}
\end{equation*}
pentru x,$\alpha,\beta \geq 0$.
unde B ( $\cdot$ ) este functia Beta, $B(x,y)=\frac{(x-1)!(y-1)!}{x+y-1)}!$ .\\
Exemplu:\\
$f(x|5,1)=\frac{x^{4}}{0.2}$\\
$f(x|2,2)=\frac{x(1-x)}{0.16}$\\\\
\includegraphics[scale=0.5]{C:/Users/Alex/Desktop/statistica/betapdf.eps}~\\[0.5cm]
{\scriptsize Cod \textbf{Matlab}:\\
\fbox{\parbox{0.8\linewidth}{%
x = [0:0.01:1];a = betapdf(x,5,1);\\b = betapdf(x,2,2);\\plot(x,y,'r',x,z,'g')}}}~\\\\
Observam ca pentru $\alpha$ mai mare, \emph{x} are probabilitatea mai mare de a lua valori mari, in timp ce pentru $\beta$ mai mare, probabilitatea ca \emph{x} sa ia valori mari este scazuta. 
\textbf{Functia de repartitie} a distributiei Beta:\\
\begin{equation*}
F(x|\alpha,\beta)=\frac{1}{B(\alpha,\beta)}\int_{0}^{x}t^{\alpha-1}(1-t)^{\beta-1}dt
\end{equation*}
\includegraphics[scale=0.5]{C:/Users/Alex/Desktop/statistica/betacdf.eps}~\\[0.5cm]
{\scriptsize Cod \textbf{Matlab}:\\
\fbox{\parbox{0.8\linewidth}{%
x = [0:0.01:1];a = betacdf(x,5,1);\\b = betacdf(x,2,2);\\plot(x,y,'r',x,z,'g')}}}~\\\\
Putem observa ca variabila $\alpha$ este responsabila forma curbei pe care functia o ia, iar variabila $\beta$ este responsabila de inaltimea functiei, astfel crescand si probabilitatea ca \emph{x} sa ia anumite valori.\\

\subsubsection{\textbf{Dependenta de parametri}}
Pentru a demonstra dependenta de parametri vom alege valori diferite pentru parametrii $\alpha$ si $\beta$ si le vom reprezenta grafic.\\
\includegraphics[scale=0.5]{C:/Users/Alex/Desktop/statistica/betaparam1.eps}~\\[0.5cm]
{\scriptsize Cod \textbf{Matlab}:\\
\fbox{\parbox{0.8\linewidth}{%
x = [0:0.01:1];a = betapdf(x,2.5,3);\\b = betapdf(x,5,3);\\c = betapdf(x,7.5,3);\\plot(x,a,'r',x,b,'g',x,c,'b')}}}~\\\\
Observam faptul ca variabila $\alpha$ controleaza dispersia densitatii, astfel pentru un $\alpha$ mai mare, x are probabilitatea mare de a lua valori dintr-un interval mai restrans.\\\\
\includegraphics[scale=0.5]{C:/Users/Alex/Desktop/statistica/betaparam2.eps}~\\[0.5cm]
{\scriptsize Cod \textbf{Matlab}:\\
\fbox{\parbox{0.8\linewidth}{%
x = [0:0.01:1];a = betapdf(x,3,2.5);\\b = betapdf(x,3,5);\\c = betapdf(x,3,7.5);\\plot(x,a,'r',x,b,'g',x,c,'b')}}}~\\\\
In acest caz, cu cat $\beta$ este mai mic, cu atat dispersia densitatii creste, \emph{x} avand probabilitatea mai mare de a lua valori dintr-un interval mai larg.\\\\
Pentru a generaliza proprietatile distributiilor vom alege valori aleatorii pentru parametrii precedenti folosind functia \emph{betarnd} si  vom reprezenta grafic dispersia acestora.\\
\includegraphics[scale=0.5]{C:/Users/Alex/Desktop/statistica/betarnd.eps}~\\[0.5cm]
{\scriptsize Cod \textbf{Matlab}:\\
\fbox{\parbox{0.8\linewidth}{%
a = betarnd(2.5,3,1,1000);\\b = betarnd(5,3,1,1000);\\c = betarnd(7.5,3,1,1000);\\d = betarnd(3,2.5,1,1000);\\e = betarnd(3,5,1,1000);\\f = betarnd(3,7.5,1,1000);\\subplot(1,2,1)\\hist(a)\\hold on\\hist(b)\\hist(c)\\subplot(1,2,2)\\hist(d)\\hold on\\hist(e)\\hist(f)}}}~\\[0.5cm]
Pentru $\alpha$ mai mare, \emph{x} ia cu o probabilitate mai mare valori mari, in timp ce pentru $\beta$ mai mic, \emph{x} are o probabilitate mai mare sa ia valori mici.\\

\section{\textbf{Distributii discrete si distributii continue}}
\subsection{\textbf{Distributia discreta Poisson}}
\textbf{Distributia Poisson}, denumita dupa matematicianul francez \textbf{Sim�on Denis Poisson}, este o distributie discreta de probabilitate care exprima probabilitatea unui numar dat de evenimente ce au loc intr-un interval fix de timp si/sau spatiu, cunoscand media acestora si timpul ultimului eveniment.\\
Densitatea de repartitie:\\
\begin{equation*}
f(x|\lambda)=\frac{\lambda^{x}}{x!}e^{-\lambda}I_{0,1,...}(x)
\end{equation*} 
pentru $\lambda,x \geq 0$, x intreg.\\
Exemplu:\\\\
$f(x|4) = \frac{4^{x}}{24}e^{-4}$\\
$f(x|10) = \frac{10^{x}}{10!}e^{-10}$\\
\includegraphics[scale=0.5]{C:/Users/Alex/Desktop/statistica/poisspdf.eps}~\\[0.5cm]
{\scriptsize Cod \textbf{Matlab}:\\
\fbox{\parbox{0.8\linewidth}{%
x=[0:1:20];\\y=poisspdf(x,4);\\z=poisspdf(x,10);\\plot(x,y,'r',x,z,'g')}}}~\\
Pe axa y se afla indicele de ocurenta, punctele fiind unite cu o linie continua doar pentru a oferi un grafic usor interpretabil. Astfel, pentru $\lambda$ mai mic, probabilitatea ca \emph{x} sa ia valori mici este ridicata, in caz contrar avand probabilitatea mare de a lua valori mai mari.\\
\textbf{Functia de repartitie}:
\begin{equation*}
F(x|\lambda)=e^{-\lambda}\sum_{i=0}^{[x]}\frac{\lambda^{i}}{i!}
\end{equation*}
\includegraphics[scale=0.5]{C:/Users/Alex/Desktop/statistica/poisscdf.eps}~\\[0.5cm]
{\scriptsize Cod \textbf{Matlab}:\\
\fbox{\parbox{0.8\linewidth}{%
x=[0:1:20];\\y=poisscdf(x,4);\\z=poisscdf(x,10);\\plot(x,y,'r',x,z,'g')}}}~\\[0.5cm]
Pentru $\lambda$ mic, \emph{x} poate lua cu probabilitate mare valori mici, in caz contrar probabilitatea fiind redusa pentru astfel de valori.\\
\subsubsection{\textbf{Dependenta de parametri}}
Pentru a reprezenta dependenta distributiei de parametrul $\lambda$, vom alege valori diferite pentru acesta si vom reprezenta grafic densitatile.\\\\ 
\includegraphics[scale=0.5]{C:/Users/Alex/Desktop/statistica/poissparam.eps}~\\[0.5cm]
{\scriptsize Cod \textbf{Matlab}:\\
\fbox{\parbox{0.8\linewidth}{%
x=[0:1:20];\\y=poisspdf(x,3);\\z=poisspdf(x,6);\\a=poisspdf(x,12);\\plot(x,y,'r',x,z,'g',x,a,'b')}}}~\\[1cm]
Forma graficelor ne indica faptul ca, cu cat variabila $\lambda$ este mai mare, cu atat dispersia e mai mare, iar variabila \emph{x} poate lua cu o probabilitate mai mare valori mari.\\
Vom alege, in continuare, valori aleatorii pentru parametrul $\lambda$ folosind functia \emph{poissrnd} si vom reprezenta grafic dispersia acestora.\\\\
\includegraphics[scale=0.5]{C:/Users/Alex/Desktop/statistica/poissrnd.eps}~\\[0.5cm]
{\scriptsize Cod \textbf{Matlab}:\\
\fbox{\parbox{0.8\linewidth}{%
x=poissrnd(3,1,1000);\\y=poissrnd(6,1,1000);\\z=poissrnd(12,1,1000);\\hist(x)\\hold on\\hist(y)\\hist(z)}}}~\\[1cm]
Se respecta proprietatile generale: pentru $\lambda$ mai mare, probabilitatea ca \emph{x} sa ia valori mai mari este ridicata.\\

\subsection{\textbf{Distributia continua Rayleigh}}
\subsubsection{\textbf{Densitatea si functia de repartitie}}
In statistica si teoria probabilitatilor, \textbf{distributia Rayleigh} este o distributie de probabilitate continua. Ea poate aparea cand un vector bidimensional are elemente ce sunt intr-o distributie normala, necorelate si cu varianta egala. Modulul vectorului va avea in acest caz o distributie Rayleigh. \textbf{Densitatea de repartitie}:\\
\begin{equation*}
f(x|\sigma)=\frac{x}{\sigma^{2}}e^{\frac{-x^{2}}{2\sigma^{2}}}
\end{equation*}
pentru $x \in [0,\infty)$ si $\sigma>0$.\\
Exemplu:\\\\
$f(x|2)=\frac{x}{4}e^{\frac{-x^{2}}{8}}$\\
$f(x|4)=\frac{x}{16}e^{\frac{-x^{2}}{32}}$\\
\includegraphics[scale=0.5]{C:/Users/Alex/Desktop/statistica/raylpdf.eps}~\\[0.5cm]
{\scriptsize Cod \textbf{Matlab}:\\
\fbox{\parbox{0.8\linewidth}{%
x=[0:0.1:20];\\y=raylpdf(x,2);\\plot(x,y,'r')}}}~\\[1cm]
Forma densitatii ne indica faptul ca \emph{x} poate lua cu o probabilitate mare valori mici, in timp ce pentru valorile mari probabilitatea este redusa.\\
\textbf{Functia de repartitie} pentru distributia Rayleigh:\\
\begin{equation*}
F(x|b)=\int_{0}^{x}\frac{t}{\sigma^{2}}e^{\frac{-t^{2}}{2\sigma^{2}}}dt = 1-e^{\frac{-x^{2}}{2\sigma^{2}}}
\end{equation*}
pentru $x\in[0,\infty)$.\\\\
\includegraphics[scale=0.5]{C:/Users/Alex/Desktop/statistica/raylcdf.eps}~\\[0.5cm]
{\scriptsize Cod \textbf{Matlab}:\\
\fbox{\parbox{0.8\linewidth}{%
x=[0:0.1:20];\\y=raylcdf(x,2);\\z=raylcdf(x,4);\\plot(x,y,'r',x,z,'g')}}}~\\[0.5cm]
Observam ca functia de repartitie are o probabilitate ridicata pentru valori mari.\\
\subsubsection{\textbf{Dependenta de parametri}}
Pentru a demonstra dependeta de parametri, vom alege valori diferite pentru parametrul $\sigma$ si vom reprezenta grafic diferentele.\\\\
Vom observa faptul ca, atunci cand $\sigma$ ia valori mici, probabilitatea ca \emph{x} sa ia valori mici este ridicata. Atunci cand $\sigma$ ia valori mari, \emph{x} are probabilitatea redusa de a lua atat valori mici, cat si valori mari.\\\\
\includegraphics[scale=0.5]{C:/Users/Alex/Desktop/statistica/raylparam.eps}~\\[0.5cm]
{\scriptsize Cod \textbf{Matlab}:\\
\fbox{\parbox{0.8\linewidth}{%
x=[0:0.1:20];\\y=raylpdf(x,0.5);\\z=raylpdf(x,2.5);\\a=raylpdf(x,7);\\plot(x,y,'r',x,z,'g',x,a,'b')}}}~\\[0.5cm]
Pentru a generaliza proprietatile distributiei vom alege valori aleatorii pentru parametrul $\sigma$ si vom reprezenta grafic dispersia acestora.\\
\includegraphics[scale=0.5]{C:/Users/Alex/Desktop/statistica/raylrnd.eps}~\\[0.5cm]
{\scriptsize Cod \textbf{Matlab}:\\
\fbox{\parbox{0.8\linewidth}{%
a=raylrnd(0.5,1,1000);\\b=raylrnd(2.5,1,1000);\\c=raylrnd(7,1,1000);\\;hist(a)\\hold on\\hist(b)\\hist(c)}}}~\\[0.5cm]
	\newpage
\subsection{\textbf{Exercitii}}
Pentru fiecare subpunct, verificati proprietatile densitatii de repartitie, calculati functia de repartitie, calculati valoarea medie M(X) si varianta Var(X).\\
\subsubsection{\textbf{X $\sim$ Unif(a=1,b=5)}}
\textbf{Densitatea de repartitie:}
\begin{center}
$\omega_\xi(x) = \begin{cases}
\frac{1}{b-a}, & a \le x \le b \\
0, & \text{in rest}
\end{cases}$
\end{center}~\\[0.5cm]
$\int_{-\infty}^{\infty}f_{x}(x)dx=\int_{a}^{b}\frac{1}{b-a}dx=\frac{1}{b-a}\int_{a}^{b}dx=\frac{b-a}{b-a}=\frac{5-1}{5-1}=\frac{4}{4}=1$~\\[0.5cm]
\textbf{Functia de repartitie:}\\
\begin{itemize}
\item daca $x < a$, atunci : $F_{X}(x)=P(X\leq x) = 0$.
\item daca $a\leq x \leq b$, atunci: $F_{X}(x)=P(X\leq x)=\int_{-\infty}^{x}f_{X}(t)dt=\int_{a}^{x}\frac{1}{b-a}dt=\frac{1}{b-a}\cdot t\bigr|_a^x=(x-a)(b-a)=(x-1)(5-1)=4(x-1).$
\item daca $x>b$, atunci : $F_{X}(x)=P(X\leq x)=1$.
\end{itemize}~\\[0.5cm]
\textbf{M[X]} = $\int_{-\infty}^{\infty}xf_{X}(x)dx=\int_{a}^{b}x\frac{1}{b-a}dx=\frac{1}{b-a}\int_{a}^{b}xdx=\frac{1}{b-a}(\frac{1}{2}x^{2})\bigr|_a^b=\frac{1}{b-a}\frac{1}{2}(b^{2}-a^{2})=\frac{(b-a)(b+a)}{2(b-a)}=\frac{b+a}{2}$\\\\
\textbf{Var[X]} = $M[X^{2}]-M[X]^{2}$\\
$M[X^{2}] = \frac{b^{2}+ab+a^{2}}{3}$\\
$M[X]^{2} = \frac{b^{2}+2ab+a^{2}}{4}$\\
$Var[X]=\frac{b^{2}+ab+a^{2}}{3}-\frac{b^{2}+2ab+a^{2}}{4}=\frac{4b^{2}+4ab+fa^{2}-3b^{2}-6ab-3a^{2}}{12}=\frac{(4-3)b^{2}+(4-6)ab+(4-3)a^{2}}{12}=\frac{b^{2}-2ab+a^{2}}{12}=\frac{(b-a)^{2}}{12}$\\

\subsubsection{\textbf{X $\sim$ Exp($\lambda$ = 0.5)}}
\textbf{Densitatea de repartitie:}\\
\begin{center}
$f_x(x) = \begin{cases}
\lambda e^{-\lambda x}, & x \geq 0 \\
0, & x<0
\end{cases}$
\end{center}~\\[0.5cm]
$\int_{-\infty}^{\infty}f_{x}(x)dx=\int_{0}^{\infty}\lambda e^{-\lambda x}dx=-e^{-\lambda x}\bigr|_0^\infty=0-(-1)=1$.\\\\
$f_{x}(x|0.5)=\frac{1}{2}e^{\frac{-x}{2}}$.\\
\newpage
\textbf{Functia de repartitie:}
\begin{itemize}
\item $x<0$, atunci: $F_{x}(X)=P(X\leq x)=0$
\item $x\geq 0$, atunci: $F_{x}(X)=P(X\leq x)=\int_{0}^{-\infty}{x}f_{x}(t)dt=\int_{0}^{x}\lambda e^{-\lambda t}dt=-e^{-\lambda t}\bigr|_0^x=1-e^{-\lambda x}$\\\\
\end{itemize}~\\[0.5cm]
$F_{x}(X|0.5)=1-e^{\frac{-x}{2}}$\\\\
\textbf{M[X]} = $\int_{0}^{\infty}xf(x)dx=\int_{0}^{\infty}x\lambda e^{-\lambda x}dx=(-xe^{-\lambda x})\bigr|_o^\infty + \int_{0}^{\infty}e^{-\lambda x}dx=(0-0)+(-\frac{1}{\lambda}e^{-\lambda x})\bigr|_0^\infty=0+(0+\frac{1}{\lambda})=\frac{1}{\lambda}$\\\\
$M[X|0.5]=\frac{1}{\frac{1}{2}} = 2$.\\\\
Var[X] = $\int_{0}^{\infty}x^{2}\lambda e^{-\lambda x}dx - \frac{1}{\lambda ^{2}} = \frac{2}{\lambda^{2}}-\frac{1}{\lambda^{2}}=\frac{1}{\lambda^{2}}$\\
Var[X$\vert$0.5]=$\frac{1}{\frac{1}{4}}=4$.

\subsubsection{\textbf{X $\sim$ Norm(m=0, $\sigma=1$)}}
\textbf{Densitatea de repartitie:}\\
\begin{center}
$f_X(x)=\frac{1}{\sigma\sqrt{2\pi}}e^{-\frac{x-m^{2}}{2\sigma^{2}}}$
\end{center}~\\[0.5cm]
$f_X(x\vert0,1)=\frac{1}{\sqrt{2\pi}}e^{-\frac{x-1}{2}}$.\\\\

$\int_{-\infty}^{\infty}f_X(x)dx=(2\pi)^{-1/2)}\int_{-\infty}^{\infty}e^{-\frac{1}{2}x^{2}}dx=(2\pi)^{-1/2}\int_{0}^{\infty}e^{-\frac{1}{2}x^{2}}dx=\\(2\pi)^{-1/2}2\left(\int_{0}^{\infty}e^{-\frac{1}{2}x^{2}}dx\int_{0}^{\infty}e^{-\frac{1}{2}x^{2}}dx\right)^{1/2}=(2\pi)^{-1/2}2\left(\int_{0}^{\infty}e^{-\frac{1}{2}x^{2}}dx\int_{0}^{\infty}e^{-\frac{1}{2}y^{2}}dy\right)^{1/2}=\\(2\pi)^{-1/2}\left(\int_{0}^{\infty}\int_{0}^{\infty}e^{-\frac{1}{2}(x^{2}+y^{2})}dydx\right)^{1/2}=(2\pi)^{-1/2}2\left(\int_{0}^{\infty}\int_{0}^{\infty}e^{-\frac{1}{2}(x^{2}+s^{2}x^{2})}xdsdx\right)^{1/2}=\\(2\pi)^{-1/2}2\left(\int_{0}^{\infty}\int_{0}^{\infty}e^{-\frac{1}{2}x^{2}(1+s^{2})}xdxds\right)^{1/2}=(2\pi)^{-1/2}2\left(\int_{0}^{\infty}\left[-\frac{1}{1+s^{2}}e^{-\frac{1}{2}x^{2}(1+s^{2})}\right]\bigr|_0^\infty ds\right)^{1/2}=\\(2\pi)^{-1/2}2\left(\int_{0}^{\infty}\left(0+\frac{1}{1+s^{2}}\right)ds\right)^{1/2}=(2\pi)^{-1/2}2\left(\int_0^\infty\frac{1}{1+s^{2}}ds\right)^{1/2}=(2\pi)^{-1/2}2(arctan(s)\bigr|_0^\infty)^{1/2}=(2\pi)^(-1/2)2(arctan(\infty)-arctan(0))^{1/2}=(2\pi)^{-1/2}2\left(\frac{\pi}{2}-0\right)^{1/2}=2^{-1/2}\pi^{-1/2}2\pi{-1/2}2{-1/2}=1$.\\\\
\textbf{Functia de repartitie:}\\
\begin {center}
$F_X(x)=\int_{-\infty}^{x}f_X(t)dt$
\end{center}~\\
Integrala nu poate fi exprimata in functii elementare, astfel fiind nevoie de un tabel de valori pentru a calcula functia.\\\\
\textbf{M[X]} = $\int_{-\infty}^{\infty}xf_{X}(x)dx=(2\pi)^{-1/2}\int_{-\infty}^{\infty}xe^{-\frac{1}{2}x^{2}}dx=(2\pi)^{-1/2}\int_{-\infty}^{0}xe^{-\frac{1}{2}x^{2}}x+(2\pi)^{-1/2}\int_{0}^{\infty}xe^{-\frac{1}{2}x^{2}}dx=(2\pi)^{-1/2}(-e^{-\frac{1}{2}x^{2}}\bigr|_{-\infty}^0+(2\pi)^{-1/2}(-e^{-\frac{1}{2}x^{2}}\bigr|_0^\infty=(2\pi)^{-1/2}-(2\pi)^{-1/2}=0$\\\\
\textbf{Var[X]} = $M[X^{2}]-M[X]^{2} = \int_{-\infty}^{\infty}f_{X}(x)dx - 0^{2} = 1-0 = 1$\\\\

\subsubsection{\textbf{X $\sim$ Binomial(n=1, p=5)}}
\textbf{Densitatea de repartitie:}\\
\begin{center}
$p_X(x) = \begin{cases}
\binom{n}{x}p^{x}(1-p)^{n-x}, & x \in \{0,1,...,n\} \\
0, & \text{in rest}
\end{cases}$
unde $\binom{n}{x}=\frac{n!}{x!(n-x)!}$ este coeficientul binomial.
\end{center}~\\\\
$p_X(x)=\binom{1}{x}5^{x}(-4)^{1-x}=\frac{1}{x!(1-x)!}5^{x}(-4)^{1-x}$.\\\\
$\sum_{x\in R_x}p_X(x)=\sum_{x=0}^{n}\binom{n}{x}p^{x}(1-p)^{n-x}=[p+(1-p)]^{n}=1^{n}=1$\\
unde am folosit formula expansiunii binomiale: $(a+b)^{n}=\sum_{x=0}^{n}\binom{n}{x}a^{x}b^{n-x}$.\\\\
\textbf{M[X]}=$M\left[\sum_{i=1}^{n}Y_i\right]=\sum_{i=1}^{n}M[Y_i]=\sum_{i=1}^{n}p=np$.\\\\
M[X$\vert$1,5] = $1\cdot 5 = 5$\\\\
\textbf{Var[X]}=$Var\left[\sum_{i=1}^{n}Y_i\right]=\sum_{i=1}^{n}Var[Y_i]=\sum_{i=1}^{n}p(1-p)=np(1-p)$\\\\
Var[X $\vert$ 1,5]=$1 \cdot 5 (1-5) = -20$\\\\

\subsubsection{\textbf{X $\sim$ Poisson($\lambda = 4)$}}
\textbf{Densitatea de repartitie:}\\
\begin{center}
$p_X(x) = \begin{cases}
e^{-\lambda}\frac{1}{x!}\lambda^{x}, & x \in \mathbb{Z} \\
0, & \text{in rest}
\end{cases}$
\end{center}~\\\\
$p_X(x)=e^{-4}\frac{1}{x!}4^{x}$\\\\

$P(X\geq x)=P(\tau_1+...+\tau_x \leq 1)$\\
$P(X\geq x)=1-P(X<x)=1-P(x\leq x-1)=1-\sum_{j=0}^{x-1}P(X=j)=1-\sum_{j=0}{x-1}p_X(j)=1-\sum_{j=0}^{x-1}\frac{\lambda^{j}}{j!}e^{-\lambda}=P(\tau_1+...+\tau_x \leq 1)$.
\newpage
\textbf{Functia de repartitie:}\\
\begin{center}
$F_X(x) = \begin{cases}
e^{-\lambda}\sum_{s=0}^{[x]}\frac{1}{s!}\lambda^{s}, & x \geq 0 \\
0, & \text{in rest}
\end{cases}$
\end{center}~\\\\
$F_X(x\vert 4)=e^{-4}\sum_{s=0}^{[x]}\frac{1}{s!}4^{s}$.\\\\

$F_X(x)=P(X\leq x)=\sum_{s=0}^{[x]}P(X=s)=\sum_{s=0}^{[x]}p_X(s)=\sum_{s=0}^{[x]}e^{-\lambda}\frac{1}{s!}\lambda^{s}=e^{-\lambda}\sum_{s=0}^{[x]}\frac{1}{s!}\lambda^{s}$\\\\

\textbf{M[X]} = $\sum_{x\in R_x}xp_X(x)=\sum_{x=0}^{\infty}xe^{-\lambda}\frac{1}{x!}\lambda^{x}=0+\sum_{x=1}^{\infty}xe^{-\lambda}\frac{1}{x!}\lambda^{x}=\sum_{y=0}^{\infty}(y+1)e^{-\lambda}\frac{1}{(y+1)!}\lambda^{y+1}=\sum_{y=0}^{\infty}(y+1)e^{-\lambda}\frac{1}{(y+1)y!}\lambda\lambda^{y}=\lambda\sum_{y=0}^{\infty}e^{-\lambda}\frac{1}{y!}\lambda^{y}=\lambda\sum_{y=0}^{\infty}p_y(y)=\lambda$.\\\\
M(X$\vert$ 4)= 4.\\\\

\textbf{Var[X]}=$M[X^{2}]-M[X]^{2}=\lambda^{2}+\lambda-\lambda^{2}=\lambda$.\\\\
Var[X$\vert$4] = 4. \\

 


\end{document}

